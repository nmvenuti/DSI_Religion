%
% File coling2016.tex
%
% Contact: mutiyama@nict.go.jp
%%
%% Based on the style files for COLING-2014, which were, in turn,
%% Based on the style files for ACL-2014, which were, in turn,
%% Based on the style files for ACL-2013, which were, in turn,
%% Based on the style files for ACL-2012, which were, in turn,
%% based on the style files for ACL-2011, which were, in turn, 
%% based on the style files for ACL-2010, which were, in turn, 
%% based on the style files for ACL-IJCNLP-2009, which were, in turn,
%% based on the style files for EACL-2009 and IJCNLP-2008...

%% Based on the style files for EACL 2006 by 
%%e.agirre@ehu.es or Sergi.Balari@uab.es
%% and that of ACL 08 by Joakim Nivre and Noah Smith

\documentclass[11pt]{article}
\usepackage{coling2016}
\usepackage{times}
\usepackage{url}
\usepackage{latexsym}
% \usepackage{cases} % Added by Author
\usepackage{amsmath} % Added by Author

%\setlength\titlebox{5cm}

% You can expand the titlebox if you need extra space
% to show all the authors. Please do not make the titlebox
% smaller than 5cm (the original size); we will check this
% in the camera-ready version and ask you to change it back.


\title{Optimization of Modeling Approach for Predicting Tolerance Level of Religious Discourse}

% Inserted from Style File for Authors from same organization
%\author{Nick Venuti \and Hope McIntyre \and Donald Brown \\
%         University of Virginia \\ ... \\ Address line}

\author{Nick Venuti \\
  University of Virginia \\
  Affiliation / Address line 2 \\
  Affiliation / Address line 3 \\
  {\tt nmv7de@virginia.edu} \\\And
  Hope McIntyre \\
  University of Virginia \\
  Affiliation / Address line 2 \\
  Affiliation / Address line 3 \\
  {\tt hm7zg@virginia.edu} \\\And
  Don Brown \\
  University of Virginia \\
  Affiliation / Address line 2 \\
  Affiliation / Address line 3 \\
  {\tt deb@virginia.edu} \\}

\date{}

\begin{document}
\maketitle
\begin{abstract}

Religious violence is one of the biggest and most complicated problems facing the world today. The number of incidents has been increasing in recent years and, unfortunately, scalable and accurate systems to predict which groups are likely to engage in such actions are not keeping pace. Additionally, this problem is compounded by lingual and cultural differences, which limit the effectiveness of understanding how tolerant or intolerant a group is without bias. To circumvent this challenge, recent studies indicate promise in the analysis of the performative character of discourse (how words are used) to estimate the tolerance level, rather than using the semantic or emotive character of text (what the words mean or imply). Using expert estimates of linguistic flexibility, a representation of the performative character of text, and thus also predictive of a text?s tolerance level, this paper describes (a) new approaches to automating the quantification of the performative character of words and (b) the predictive efficacy of these approaches versus traditional semantic indicators of tolerance or intolerance. To implement the pipeline, a judgment identifier was developed along with multiple semantic density algorithms to extract the frequency of judgments and flexibility of keyword contexts, respectively. Test results show that text mining algorithms can accurately estimate the language flexibility of religious discourse. These results provide evidence that the performative characteristics of language better predict tolerance level than the semantic characteristics of language.

\end{abstract}

\section{Introduction}
\label{intro}

[Insert Text Here]

%
% The following footnote without marker is needed for the camera-ready
% version of the paper.
% Comment out the instructions (first text) and uncomment the 8 lines
% under "final paper" for your variant of English.
% 
\blfootnote{
    %
    % for review submission
    %
    % \hspace{-0.65cm}  % space normally used by the marker
    % Place licence statement here for the camera-ready version, see
    % Section~\ref{licence} of the instructions for preparing a
    % manuscript.
    %
    % % final paper: en-uk version (to license, a licence)
    %
    % \hspace{-0.65cm}  % space normally used by the marker
    % This work is licensed under a Creative Commons 
    % Attribution 4.0 International Licence.
    % Licence details:
    % \url{http://creativecommons.org/licenses/by/4.0/}
    % 
    % final paper: en-us version (to licence, a license)
    
    \hspace{-0.65cm}  % space normally used by the marker
    This work is licenced under a Creative Commons 
    Attribution 4.0 International License.
    License details:
    \url{http://creativecommons.org/licenses/by/4.0/}
}

\section{Literature Review}
\label{lit review}

[Insert Text Here]

\section{Data}
\label{data}

[Insert Text Here]

\begin{table}
\begin{tabular}{| l | l | l | l |}
\hline
    Group & Rank & Affiliation & No. of Docs. \\
  \hline
  \hline			
Westboro Baptist & 1 & Baptist & 419 \\
\hline
Faithful Word Baptist Church & 2 & Baptist & 228 \\
\hline
Nouman Ali Khan & 3 & Sunni Muslim & 88 \\
\hline
Dorothy Day & 4 & Catholic & 774 \\
\hline
John Piper & 4 & Baptist & 579 \\
\hline
Steve Shepherd & 4 & Christian & 728 \\
\hline
Rabbinic Texts & 6 & Jewish & 166 \\
\hline
Unitarian texts & 7 & Unitarian & 276 \\
\hline
Meher Baba & 8 & Spiritualist & 265 \\
\hline
\end{tabular}
\caption{Data Sources}
\end{table}

\section{Signals}
\label{signals}

[Insert Text Here]

\subsection{Sentiment}
\label{sect:sentiment}


\subsection{Judgment Numbers}
\label{sect:judgments}


\subsection{Context Vectors}
\label{sect:cv}

\begin{equation}
V_{i}=[x_{i1},x_{i2},...,x_{iv}]
\end{equation}

\begin{equation}
c_{ij}=\sum_{k \epsilon window(w_{i})} dsm(w)
\end{equation}

\begin{equation}
semantic density(w) = \sum_{k=1}^{l-1} \frac{<c_{ik},c_{i(k+1)}>}{||c_{ik}||*||c_{i(k+1)}||}
\end{equation}


\subsection{Network Quantification}
\label{sect:network}

\section{Hyperparameter Optimization}
\label{hyper}

[Insert Text Here]

\subsection{Co-Occurrence Window}
\label{cooc window}


\subsection{Context Window}
\label{sect:context window}


\subsection{Keyword Rank Window}
\label{sect:keyword}


\subsection{Network Adjacency Angle}
\label{sect:angle}

\section{Results}
\label{results}

% acc(bin, model) = \{1, |\hat{y}-y|\}

\begin{equation}
acc(bin, model) = 
\begin{cases} 
1, & \text{if } |\hat{y}-y| \leq 1 \\
0, & \text{otherwise}
\end{cases}
\end{equation}

\begin{equation}
acc(model) = \frac{1}{|bins|} \sum_{bin  \epsilon  bins} acc(bin, model)
\end{equation}


[Insert Text Here]

\section{Conclusions}
\label{conclusions}

[Insert Text Here]


% include your own bib file like this:
%\bibliographystyle{acl}
%\bibliography{coling2016}

\begin{thebibliography}{}

\bibitem[\protect\citename{Aho and Ullman}1972]{Aho:72}
Alfred~V. Aho and Jeffrey~D. Ullman.
\newblock 1972.
\newblock {\em The Theory of Parsing, Translation and Compiling}, volume~1.
\newblock Prentice-{Hall}, Englewood Cliffs, NJ.

\bibitem[\protect\citename{{American Psychological Association}}1983]{APA:83}
{American Psychological Association}.
\newblock 1983.
\newblock {\em Publications Manual}.
\newblock American Psychological Association, Washington, DC.

\bibitem[\protect\citename{{Association for Computing Machinery}}1983]{ACM:83}
{Association for Computing Machinery}.
\newblock 1983.
\newblock {\em Computing Reviews}, 24(11):503--512.

\bibitem[\protect\citename{Chandra \bgroup et al.\egroup }1981]{Chandra:81}
Ashok~K. Chandra, Dexter~C. Kozen, and Larry~J. Stockmeyer.
\newblock 1981.
\newblock Alternation.
\newblock {\em Journal of the Association for Computing Machinery},
  28(1):114--133.

\bibitem[\protect\citename{Gusfield}1997]{Gusfield:97}
Dan Gusfield.
\newblock 1997.
\newblock {\em Algorithms on Strings, Trees and Sequences}.
\newblock Cambridge University Press, Cambridge, UK.

\end{thebibliography}

\end{document}
